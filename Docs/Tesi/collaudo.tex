Ogni caso d'uso può dar luogo ad una varietà di comportamenti in base alla sequenza di percorsi base ed alternativi che viene eseguita.
Per ogni caso d'uso è possibile definire un insieme di casi di test che esercitino tutti i possibili comportamenti. La dizione  ``tutti i comportamenti'' è di per sè vaga, e deve essere chiarita specificando il tipo di 
copertura (tutt gli gli archi, tutti i percorsi, etc.) che si intende usare. Comunque scelto il tipo di copertura i casi di test derivati possono essere un numero molto elevato. Per ridurre l'ordine di grandezza di questo numero
si possono definire degli scenari di alto livello che aggreghino più casi di test insieme. In seguito utilizziamo questa tecnica per effettuare un collaudo dell'applicativo.

\section{Scenari di alto livello}
Si elencano gli scenari che sono stati usato per il collaudo.

\begin{enumerate}
 \item L'Operatore inserisce una convenzione con una nuova dittà specificando una particolare ripartizione fra il personale definendo anche la prima rata. \label{MS1}
 \item L'Operatore modifica una rata di un contributo \label{MS2}
 \item L'Operatore modifica i dati di una ditta \label{MS3}
 \item L'Operatore inserisce un contributo con allegati \label{MS4}
 \item Il Docente inserisce la documentazione relativa ad una convenzione \label{MS5}
 \item L'Amministratore aggiunge un utente \label{MS6}
\end{enumerate}

In tabella \ref{macro_scenari} i casi di test esercitati per ogni scenario.\\


\begin{table}[h]
\label{macro_scenari}
\caption{Casi di test esercitati per scenario}
\centering
  \begin{tabular}{| c | c |} 
    \hline
    Numero di scenario & Casi di test esercitati \\
    \hline
    \ref{MS1} & \ref{UC_new_contract} base, \ref{UC_new_company} alt3, \ref{UC_new_installment}\\
    \hline
    \ref{MS2} & \ref{UC_view_contract_list} base, \ref{UC_edit_contract} base, \ref{UC_edit_installment} \\
    \hline
    \ref{MS3} & \ref{UC_view_company_list} base, \ref{UC_edit_company} base \\
    \hline
    \ref{MS4} & \ref{UC_new_contract}\\
    \hline
    \ref{MS5} & \ref{UC_view_own_contract_list} base, \ref{UC_manage_attachments} base\\
    \hline
    \ref{MS6} & \ref{UC_new_user}\\
    \hline
  \end{tabular} 

\end{table} 

\section{Collaudo}

\begin{itemize}
 \item Scenario \ref{MS1}\\\\
 {
 \footnotesize
  \begin{longtable}{|c|p{3cm}|p{3cm}|p{3cm}|c|}
    \hline
    Passo & Descrizione sequenza operazioni & Risultato atteso & Risultato ottenuto & Ok\\
    \hline
    1 & L'Operatore clicca sul pulsante ``Crea Convenzione'' & Viene visualizzata una schermata contenente il primo passo della procedura di creazione nel quale si possono inserire i dati generali della convenzione& Uguale 
      al risultato atteso& Sì\\
    \hline
    2 & L'Operatore clicca sul pulsante ``Aggiungi'' di fianco al campo dittà` & Viene visualizzata una finestra di dialogo contenente i dati della dittà da inserire & Uguale al risultato atteso & Sì\\
    \hline
    3 & L'Operatore inserisce i dati e clicca su ``Salva'' & Si ritorna al primo passo della procedura, la dittà appena creata è selezionata automaticamente nel menù a tendina & uguale al risultato atteso & Sì\\
    \hline
    4 & L'Operatore clicca next e passa al secondo passo della procedura, quindi inserisce una quota per il personale, infine aggiunge uno o più docenti tramite l'apposito pulsante ``Aggiungi'' in basso, in modo che la somma sia 100 & La schermata
      riflette le scelte dell'operatore aggiornando i valori percentuali dei vari campi. & Uguale al risultato ottenuto& Sì\\
    \hline
    5 & L'operatore clicca su ``Avanti'',quindi sul pulsante ``Aggiungi''nel passo successivo della procedura & Si apre una finestra di dialogo contenente la procedura di creazione di una rata & Uguale a quello atteso & Sì\\
    \hline
    6  & L'Operatore inserisce i campi per la rata e completa la procedura cliccando su ``Salva'' & La finestra di dialogo si chiude e nella schermata sottostatnte compare la rata appena inserita & Uguale a quello atteso & Sì\\
    \hline
    7 & L'Operatore clicca su ``Avanti'' due volte (saltando il passo relativo agli allegati) quindi clicca su``Salva'' & Si torna alla pagina iniziale & Uguale a quello atteso & Sì \\
    \hline
    8 & L'Operatore clicca su ``Visualizza elenco contratti'' & Si apre un elenco di contratti fra cui è possibile ritrovare la convenzione appena inserita & Uguale a quello atteso & Sì\\
    \hline
\end{longtable}
}

Qui gli screenshots!!!Bellini e nn malsani!

\end{itemize}
