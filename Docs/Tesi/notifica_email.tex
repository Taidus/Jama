Nell'applicazione ricopre un ruolo cruciale la notifica tramite e-mail al Docente di ciò che riguarda contratti da lui stipulati.\\
Qui di seguito viene illustrata la struttura del codice che viene usato per questo procedimento e un estratto di template:

\begin{lstlisting}
public void notifyEvent(Contract c) {
	
	TemplateFiller filler = new TemplateFiller(c, "email@address.com");
	StringWriter out = new StringWriter();

	Template template = Config.fmconf.getTemplate(Config.templateFileName);
	temp.process(filler, out);
	String mailContent = out.toString();

	User u = userDao.getBySerialNumber(c.getChief().getSerialNumber());
	String email = u.getEmail();

	send(email, "Title", mailContent);
}

private void send(String recipientEmail, String subject, String text) {

	String host = "smtp.gmail.com";
	String username = "jama.mail.services";
	String password = "password";

	MimeMessage message = new MimeMessage(mailSession);
	try {

		message.setRecipient(RecipientType.TO, new InternetAddress(recipientEmail));
		message.setSubject(subject);
		message.setText(text);
		message.saveChanges();

		Transport t = mailSession.getTransport("smtps");
		try {
			t.connect(host, username, password);
			t.sendMessage(message, message.getAllRecipients());
		} finally {
			t.close();
		}

	} catch (MessagingException e) {
		FacesContext context = FacesContext.getCurrentInstance();
		context.addMessage(null, new FacesMessage(Messages.getString("err_sendingMail")));
	}
}

public class TemplateFiller {
	private Contract contract;
	private String mail;


	public TemplateFiller(Contract contract, String mail) {
		super();
		this.contract = contract;
		this.mail = mail;
	}


	public Contract getContract() {
		return contract;
	}


	public String getMail() {
		return mail;
	}

}
\end{lstlisting}

\begin{lstlisting}
Gent.mo Professore,
questa comunicazione riguarda la convenzione dal titolo "${contract.title}" fra ${contract.department.name} e ${contract.company.name} [...]
\end{lstlisting}

Per produrre il contenuto della mail viene usata la libreria FreeMarker. Si è già spiegato il suo funzionamento nel paragrafo \ref{freemarker}; qui si può vedere la sua applicazione all'operazione di notifica.\\
La configurazione di FreeMarker viene definita una sola volta all'avvio, ed è accessibile tramite la classe \lstinline{util.Config} (che contiene la configurazione di Jama). La classe \lstinline{Config} espone inoltre anche gli attributi che contengono i nomi dei file di template.\\
Definita la configurazione è possibile passare al caricamento del template e al suo utilizzo. Tale operazione viene eseguita nel metodo \lstinline{notifyEvent} tramite la chiamata al metodo \lstinline{getTemplate}. Quest'ultimo viene poi riempito utilizzando il bean \lstinline{TemplateFiller} e subito dopo viene prodotto l'output. \\
A questo punto viene chiamato il metodo \lstinline{send}, che è quello che si occupa dell'invio vero e proprio della posta elettronica. Per fare ciò, vengono semplicemente adoperate le API Java del package \lstinline{javax.mail}, usando gli opportuni parametri dipendenti dall'applicazione (come ad esempio l'host o il protocollo).\\
L'intero processo viene eseguito dalla classe \lstinline{MailSender}, che si trova nel package \lstinline{Util}.
