Nell'applicazione ricopre un ruolo cruciale la notifica ai Docenti di ciò che riguarda contratti da lui stipulati, che avviene tramite e-mail.\\
Qui di seguito viene illustrata la struttura del codice che viene usato per questo procedimento e un esempio di template:

\begin{lstlisting}
public void notifyEvent(Contract c) {
	
	TemplateFiller filler = new TemplateFiller(c, "email@address.com");
	StringWriter out = new StringWriter();

	Template template = Config.fmconf.getTemplate(Config.templateFileName);
	temp.process(filler, out);
	String mailContent = out.toString();

	User u = userDao.getBySerialNumber(c.getChief().getSerialNumber());
	String email = u.getEmail();

	send(email, "Title", mailContent);
}

private void send(String recipientEmail, String subject, String text) {

	String host = "smtp.gmail.com";
	String username = "jama.mail.services";
	String password = "password";

	MimeMessage message = new MimeMessage(mailSession);
	try {

		message.setRecipient(RecipientType.TO, new InternetAddress(recipientEmail));
		message.setSubject(subject);
		message.setText(text);
		message.saveChanges();

		Transport t = mailSession.getTransport("smtps");
		try {
			t.connect(host, username, password);
			t.sendMessage(message, message.getAllRecipients());
		} finally {
			t.close();
		}

	} catch (MessagingException e) {
		FacesContext context = FacesContext.getCurrentInstance();
		context.addMessage(null, new FacesMessage(Messages.getString("err_sendingMail")));
	}
}

public class TemplateFiller {
	private Contract contract;
	private String mail;


	public TemplateFiller(Contract contract, String mail) {
		super();
		this.contract = contract;
		this.mail = mail;
	}


	public Contract getContract() {
		return contract;
	}


	public String getMail() {
		return mail;
	}

}
\end{lstlisting}

