L'applicazione Jama si occupa della gestione delle Convenzioni. Una convenzione è un contratto tra un Responsabile Scientifico - tipicamente un docente - ed una azienda per fare ricerca scientifica su un argomento concordato. In particolare, Jama serve a tenere traccia dello stato di tutte le convenzioni presenti e passate all'interno del \textsl{Dipartimento di Ingegneria dell'Informazione}, con una possibile futura estensione ad altri dipartimenti dell'Ateneo.\newline
Questo capitolo, come si può intuire, è un'introduzione al resto del documento, che descrive brevemente il contenuto dei prossimi capitoli.\newline
Per prima cosa si descrive il "dietro le quinte" del problema, ovvero il processo di definizione di una convenzione: i primi contatti tra un docente ed un'azienda, l'approvazione da parte del Consiglio, e tutto ciò che ne consegue; verranno inoltre descritti brevemente i requisiti dell'applicazione, prima di passare alla vera descrizione del lavoro degli autori: com'è fatta l'applicazione, come si usa, e anche qualche nozione sugli strumenti usati per arrivare al risultato finale.

\section{Processo di definizione di una Convenzione}
La descrizione del processo è tratta da un documento ufficiale dell'\textsl{Università degli Studi di Firenze}\footnote{\url{http://www.polobiotec.unifi.it/upload/sub/att_commerciale/att_commerciale.pdf}}. Quella che segue è una descrizione più informale.

\paragraph{Livello Politico Gestionale}
Questo paragrafo descrive il processo di proposizione e di approvazione di una convenzione.\newline
\begin{enumerate}
\item Il Responsabile Scientifico prende contatto con un'azienda per accordarsi sui termini dell'eventuale convenzione che li legherà.
\item
Il Responsabile Scientifico, insieme al RAS, valuta i costi e stila la Tabella di Ripartizione dei compensi.
\item Il RAS prepara un documento da discutere nel prossimo Consiglio di Dipartimento, che deciderà se approvare o meno la convenzione.
\item In caso di approvazione da parte del consiglio, l'UAS procede all'inserimento della stessa in CIA e in Jama.\newpage
\end{enumerate}
\paragraph{Livello Gestionale}
Il Livello Gestionale descrive alcuni requisiti dell'applicazione.\newline
\begin{enumerate}
\item Una settimana prima della scadenza di ogni rata della convenzione, Jama provvede ad inviare una mail al Responsabile Scientifico, che dovrà indicare se procedere o meno all'emissione della nota di debito - inviando una mail alla struttura amministrativa adatta.
\item In caso il Responsabile Scientifico dia il suo benestare, l'amministrazione emette la nota di debito e la registra su Jama. In caso contrario, l'UAS prevede a modificare la scadenza della rata su Jama.
\item All'emissione della fattura di una rata, l'Amministrazione provvede a registrarla in Jama.\newline
\end{enumerate}

\section{Requisiti}

Il precedente paragrafo introduce alcuni dei requisiti principali dell'applicazione: come già detto, Jama si occupa della gestione delle convenzioni. Quello che non è stato ancora detto è che si occupa anche dei \textsl{Contributi}: un contributo è molto simile ad una convenzione, le uniche differenze sono che non è ammessa una ripartizione ad un personale, non c'è fattura ma solo ricevuta, non c'è IVA, la tassazione di Ateneo ha aliquote diverse.\newline
Dopo questa doverosa precisazione, si elencano i requisiti di Jama:

\begin{itemize}
\item inserire una convenzione/contributo
\item visualizzare l'elenco delle convenzioni/contributi
\item visualizzazione dello scadenziario
\item gestione delle rate di una convenzione/contributo (inserimento, modifica, eliminazione, visualizzazione)
\item modifica di una convenzione/contributo
\item eliminazione di una convenzione/contributo
\item visualizzazione di una convenzione/contributo\newline
\end{itemize}
Il prossimo capitolo formalizzerà meglio i requisiti precedenti, attraverso la descrizione dei \textsl{Casi d'Uso} dell'applicazione, ovvero delle funzionalità implementate e del modo in cui un utente possa utilizzarle.

\section{Analisi}
In questo capitolo verrà descritto il lavoro iniziale da fare quando si sviluppa un'applicazione: un \textsl{Modello di Dominio} e una descrizione dei \textsl{Casi d'Uso}.
\paragraph{Modello di Dominio}
Il Modello di Dominio è una descrizione delle relazioni tra le \textsl{classi} dell'applicazione: senza addentrarsi nei dettagli della \textsl{Programmazione ad Oggetti}, il concetto è che una classe rappresenta un oggetto del mondo reale; così una \textsl{Convenzione} diventa un \textsl{Agreement} nella nostra applicazione, come un \textsl{Responsabile Scientifico} diventa uno \textsl{ChiefScientist}.\newline 
I termini del Modello di Dominio sono in inglese, come si usa fare nell'Ingegneria del Software, perciò di solito si stila un \textsl{Dizionario} che associa termini del dominio a termini del modello: ad esempio, nel dizionario si scriverà qualcosa tipo ''Convenzione = Agreement'', così da non perdersi nelle eventuali ambiguità provocate dall'uso di due lingue diverse.\newline
Quest'ultimo aspetto è una delle caratteristiche più utili di un modello di dominio: specifica un insieme di termini non ambigui che si riferiscono al dominio. In questo modo tutte le volte che ci si riferirà ad un Responsabile Scientifico si scriverà "Responsabile Scientifico", e non "Docente" per esempio, così che non ci sia modo di confondersi. Sembra un problema di poco conto, ma è fondamentale mettersi d'accordo su come ci si riferisce ad uno stesso concetto, per evitare che il caos prenda il controllo e regni sovrano.
\paragraph{Casi d'Uso}
Il diagramma dei Casi d'Uso descrive l'insieme delle funzionalità dell'applicazione, mettendo in evidenza come usarle e chi le usa. In questo modo, per esempio, si specifica che è un Operatore che si occupa dell'\textsl{Inserimento di una Convenzione/Contributo}, e non un Responsabile Scientifico.\newline
Il diagramma dei Casi d'Uso è una specifica di alto livello di ciò che l'applicazione deve fare, ed è per questo che è fondamentale per lo sviluppo iniziale di un'applicazione: gli sviluppatori e gli \textsl{Stakeholders} - chi ha interesse nello sviluppo dell'applicazione - si possono mettere d'accordo su cosa l'applicazione deve fare, e come lo deve fare, cioè come l'utente deve interagire con il sistema; spesso quest'ultimo punto viene specificato in termini di \textsl{Mockups}, ovvero di disegni che prototipizzino quella che diventerà l'interfaccia utente.
\section{Tecnologie}
L'applicazione è stata sviluppata in \textsl{Java}, con l'aiuto di alcuni \textit{framework} che gestiscono vari aspetti dell'architettura. In particolare, \textsl{JPA} - \textsl{Java Persistence API} all'anagrafe - si occupa, insieme ad \textsl{Hibernate}, della persistenza - un problema ricorrente nello sviluppo di applicazioni \textit{Enterprise}  che verrà discusso meglio nel paragrafo dedicato; \textsl{Java Server Faces}, o \textsl{JSF}, si occupa del cosiddetto \textsl{Presentation Layer} dell'applicazione, ovvero dell'interfaccia utente; \textsl{CDI}, ovvero \textsl{Context and Dependency Injection}, è un grande aiuto nello sviluppo generale dell'applicazione, permettendo di gestire in maniera semplice ed efficace le relazioni tra i vari componenti dell'applicazione; infine \textsl{Deltaspike} si occupa della gestione dei ruoli e dei permessi degli utenti dell'applicazione.\newline
Ognuno dei \textit{framework} sopra menzionati verrà discusso in dettaglio  nel Capitolo ~\ref{chapter:Tecnologie}, che descriverà meglio il loro ruolo, i vantaggi ottenuti dal loro uso e il modo con cui usarli.
\section{Utilizzo}
Come è facilmente intuibile, questo capitolo mostrerà come si usa l'applicazione, descrivendo come utilizzare ognuna delle funzionalità dell'applicazione. In realtà, l'interfaccia di Jama non è complicata, ma è sempre bene far vedere in maniera esatta come si usa. Verranno inclusi \textit{screenshot} e spiegazioni dettagliate di ogni aspetto dell'applicazione: dalla creazione di una convenzione, alla sua modifica, alla visualizzazione, a tutte le altre funzionalità.
\section{Dietro le Quinte}
Questo capitolo descrive come il sistema reagisce internamente all'input dell'utente. E' pensato per chi avrà a che fare con l'ampliamento delle funzionalità dell'applicazione, per avere un'idea iniziale di come affrontare il problema. Si occuperà di aspetti non trattati altrove, ad esempio come funziona il Login e come si ottengono le convenzioni da visualizzare.
Senza entrare nei dettagli, è opportuno far notare subito che qualora le credenziali inserite nella fase di Login non vengano riconosciute consultando il database interno del'applicazione, si provvederà a contattare un database di facoltà, \textsl{SIAF}, attraverso il protocollo \textsl{LDAP} - LDAPS in realtà, cioè LDAP Secure - gestito in Jama dalla libreria OpenLDAP, che verrà descritta in un paragrafo successivo.
