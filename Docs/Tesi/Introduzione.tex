L'applicazione Jama si occupa della gestione delle Convenzioni. Una convenzione è un contratto tra un Responsabile Scientifico - tipicamente un docente - ed una azienda per fare ricerca scientifica su un argomento concordato. In particolare, Jama serve a tenere traccia dello stato di tutte le convenzioni presenti e passate all'interno del \textsl{Dipartimento di Ingegneria dell'Informazione}, con una possibile futura estensione ad altri dipartimenti dell'Ateneo.\newline
Questo capitolo descrive brevemente il contenuto dei prossimi capitoli.\newline
Per prima cosa si descrive il processo di definizione di una convenzione: i primi contatti tra un docente ed un'azienda, l'approvazione da parte del Consiglio, e tutto ciò che ne consegue; verranno inoltre descritti brevemente i requisiti dell'applicazione, prima di passare alla discussione del lavoro degli autori: com'è fatta l'applicazione, come si usa, e anche qualche nozione sugli strumenti usati per arrivare al risultato finale.

\section{Processo di definizione di una Convenzione}
La descrizione del processo è tratta da un documento ufficiale dell'\textsl{Università degli Studi di Firenze}\footnote{\url{http://www.polobiotec.unifi.it/upload/sub/att_commerciale/att_commerciale.pdf}}. Quella che segue è una descrizione più informale.

\paragraph{Livello Politico Decisionale}
Questo paragrafo descrive il processo di proposizione e di approvazione di una convenzione.\newline

\begin{enumerate}
\item Il Responsabile Scientifico prende contatto con un'azienda per accordarsi sui termini dell'eventuale convenzione che li legherà.
\item
Il Responsabile Scientifico, insieme al RAS, valuta i costi e stila la Tabella di Ripartizione dei compensi.
\item Il RAS prepara un documento da discutere nel prossimo Consiglio di Dipartimento, che deciderà se approvare o meno la convenzione.
\item In caso di approvazione da parte del consiglio, l'UAS procede all'inserimento della stessa in CIA e in Jama.
\end{enumerate}

\paragraph{Livello Gestionale}
Il Livello Gestionale descrive alcuni requisiti dell'applicazione.\newline
\begin{enumerate}
\item Una settimana prima della scadenza di ogni rata della convenzione, Jama provvede ad inviare una mail al Responsabile Scientifico, che dovrà indicare se procedere o meno all'emissione della nota di debito - inviando una mail alla struttura amministrativa adeguata.
\item In caso il Responsabile Scientifico dia il suo benestare, l'amministrazione emette la nota di debito e la registra su Jama. In caso contrario, l'UAS prevede a modificare la scadenza della rata su Jama.
\item All'emissione della fattura di una rata, l'Amministrazione provvede a registrarla in Jama.\newline
\end{enumerate}

\section{Requisiti}
Oltre alla gestione delle Convenzioni, Jama si occupa anche dei \textsl{Contributi}: un contributo si differenzia da una convenzione per il fatto che non è ammessa una ripartizione ad un personale, non c'è fattura ma solo ricevuta, non c'è IVA e la tassazione di Ateneo ha aliquote diverse.\newline
Si elencano di seguito i requisiti di Jama:

\begin{itemize}
\item inserire una convenzione/contributo
\item visualizzare l'elenco delle convenzioni/contributi
\item visualizzazione dello scadenzario
\item gestione delle rate di una convenzione/contributo (inserimento, modifica, eliminazione, visualizzazione)
\item modifica di una convenzione/contributo
\item eliminazione di una convenzione/contributo
\item visualizzazione di una convenzione/contributo\newline
\end{itemize}

Ognuno dei requisiti viene discusso in dettaglio nel seguito del documento.

