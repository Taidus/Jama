L'applicazione realizzata si occupa della gestione delle Convenzioni. Una convenzione è un contratto tra un Responsabile Scientifico
- tipicamente un docente - ed una azienda per fare ricerca scientifica su un argomento concordato.
In particolare, l'applicativo serve a monitorare lo stato di tutte le convenzioni attive o esaurite all'interno del
\textsl{Dipartimento di Ingegneria dell'Informazione}, con una possibile futura estensione ad altri dipartimenti dell'Ateneo.\newline
Per prima cosa si descrive il processo di definizione di una convenzione: i primi contatti tra un docente ed un'azienda,
l'approvazione da parte del Consiglio, e tutto ciò che ne consegue; 
vengono quindi descritti brevemente i requisiti dell'applicazione, prima di passare alla discussione del lavoro degli autori:
una analisi del problema che ha portato alla definizione di casi d'uso e modello di dominio, una ricerca delle tecnologie da utilizzare, la stesura del codice
e infine un collaudo.


\section{Processo di definizione di una Convenzione}
La descrizione del processo è tratta da un documento ufficiale dell'\textsl{Università degli Studi di Firenze}\footnote{\url{http://www.polobiotec.unifi.it/upload/sub/att_commerciale/att_commerciale.pdf}}. Quella che segue è una descrizione più informale.

\paragraph{Livello Politico Decisionale}
Questo paragrafo descrive il processo di proposizione e di approvazione di una convenzione.\newline

\begin{enumerate}
\item Il Responsabile Scientifico prende contatto con un'azienda per accordarsi sui termini dell'eventuale convenzione che li legherà.
\item
Il Responsabile Scientifico, insieme al RAS, valuta i costi e stila la Tabella di Ripartizione dei compensi.
\item Il RAS prepara un documento da discutere nel prossimo Consiglio di Dipartimento, che deciderà se approvare o meno la convenzione.
\item In caso di approvazione da parte del consiglio, l'UAS procede all'inserimento della stessa in CIA e nell'applicazione.
\end{enumerate}

\paragraph{Livello Gestionale}
Il Livello Gestionale descrive i passaggi fondamentali che intercorrono fra la creazione e la chiusura di una convenzione.\newline
\begin{enumerate}
\item Una settimana prima della scadenza di ogni rata della convenzione,l'applicativo provvede ad inviare una e-mail al Responsabile Scientifico, che dovrà indicare se procedere o meno all'emissione della nota di debito - inviando una e-mail alla struttura amministrativa adeguata.
\item In caso il Responsabile Scientifico dia il suo benestare, l'amministrazione emette la nota di debito e la registra sull'applicativo. In caso contrario, l'UAS prevede a modificare la scadenza della rata.
\item All'emissione della fattura di una rata, l'Amministrazione provvede a registrarla nell'applicativo.\newline
\end{enumerate}

\section{Requisiti}
L'obiettivo principale dell'applicativo è la gestione delle convenzioni, tuttavia deve anche occuparsi di \textsl{Contributi}: un contributo si differenzia da una convenzione per il fatto che non è ammessa ripartizione al personale, non c'è fattura ma solo ricevuta, non c'è IVA e la tassazione di Ateneo ha aliquote diverse.\newline
L'applicazione deve inoltre prevedere la possibilità di definire utenti, ciascuno con un ruolo assegnato, in modo che ogni utente abbia accesso ad un sottoinsieme specifico di azioni. Entrando un po' più in dettaglio l'applicativo deve
consentire di:

\begin{itemize}
\item inserire convenzioni e contributi
\item allegare dei documenti ad una convenzione o ad un contributo
\item eliminare una convenzione o un contributo
\item visualizzare o modificare i dati di una convenzione o di un contributo
\item visualizzare l' elenco delle convenzioni e contributi inseriti
\item consultare uno scadenzario delle rate
\item gestire, ovvero, inserire, modificare o visualizzare le rate di una convenzione o di un contributo

\item gestire diversi tipi di utente, tra cui l'Operatore (che ha il compito di inserire e aggiornare le convenzioni del dipartimento a cui afferisce) e il Docente (che può visualizzare le convenzioni di cui è il responsabile)

\item notificare agli utenti del servizio, in modo automatico, tramite e-mail, particolari eventi del ciclo di vita della convenzione (ad esempio la creazione o lo scadere di una rata)
\item integrarsi con il Sistema Informatico dell'Ateneo Fiorentino (SIAF) per ottenere dati sul personale e sui dipartimenti dell'Ateneo
\end{itemize}

Ognuno dei requisiti viene discusso in dettaglio nel seguito del documento.

