Il lavoro svolto consiste nell'analisi, nella progettazione e nello sviluppo di un' applicazione web per la gestione di dati e di processi che agiscono su questi; nello specifico, l'applicazione deve consentire la gestione delle convenzioni di ricerca e dei contributi.\\
Durante la fase di analisi è stato studiato il problema ed è stata stilata una lista di requisiti che il sistema deve rispettare.
A partire da tali requisiti, sono stati quindi definiti dei casi d'uso che descrivono completamente il comportamento dell'applicazione.\\\\
Si è poi passati alla fase di progettazione. Il primo passo è stata la decisione delle tecnologie da utilizzare. La scelta è ricaduta su \textsl{Java Platform Enterprise Edition} (o, più brevemente, \textsl{Java EE}),
una piattaforma standard nello sviluppo di applicazioni \textit{enterprise} (cioè di applicazioni che devono essere usate da imprese o organizzazioni di varia natura, e non da singoli individui).
Java EE include infatti molti componenti che risultano estremamente utili per risolvere problemi tipici di software di questo tipo, ad esempio \textsl{JPA} (\textsl{Java Persistence API}) per la persistenza dei dati,
\textsl{CDI} (\textsl{Contexts and Dependency Injection}) per collegare la logica applicativa con il livello web tramite una gestione automatica degli oggetti in relazione al contesto in cui vengono utilizzati, e \textsl{JSF} 
(\textsl{Java Server Faces}) per lo sviluppo del livello di presentazione. Questi framework sono poi stati affiancati da \textsl{Apache 
Deltaspike}, che si occupa della gestione della sicurezza, impedendo che alcune funzionalità dell'applicazione  possano essere eseguite se l'utente non possiede i permessi necessari.\\
Successivamente è stato definito il modello dell'applicazione. In realtà, i modelli definiti sono due: quello di business, che rappresenta le entità che riguardano la logica dell'applicazione, e quello relativo agli utenti,
che che modella utenti e ruoli dell'applicazione. Questi due modelli sono tra loro \textquoteleft ortogonali\textquoteright{}, nel senso che - in linea di massima - è possibile modificare l'uno senza essere costretti a modificare anche l'altro (anche se, a dire la verità, c'è un leggero accoppiamento fra i due, dovuto al fatto che alcuni utenti dell'applicazione - i docenti - sono anche entità del modello di business).\\
\\
È seguita la fase di sviluppo, in cui si è utilizzato tutto il materiale precedentemente prodotto per scrivere l'applicazione. Infine, è stato effettuato un collaudo per la verifica delle funzionalità, al termine dei quali si è passati alla fase di distribuzione (\textit{deploy}) del software.\\
\\
