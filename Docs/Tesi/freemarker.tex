
\textsl{FreeMarker} è un package Java che consente di produrre output testuale basandosi su un template. È stato progettato specificatamente per la generazione automatica di pagine HTML, anche se può creare vari tipi di pagine. La caratteristica principale è che consente di separare la logica che costruisce la pagina con il template in base alla quale viene prodotta; in questo modo, il template è modificabile senza l'intervento di un programmatore (anche se, senza modificare il codice, non è possibile utilizzare informazioni non ricavabili dagli oggetti a cui si ha già accesso).\\
Nei paragrafi che seguono verrà descritto il package FreeMarker e spiegato come è stato usato all'interno dell'applicazione. Per maggiori informazioni su questo tool, si rimanda alla pagina web relativa al progetto \cite{freemarker}, dalla quale sono stati presi - fra l'altro - gli esempi sul funzionamento e utilizzo del software.


\subsubsection{Funzionamento}
Per il funzionamento di FreeMarker sono necessari due elementi:
\begin{itemize}
\item un file di template
\item i parametri con cui \textquotedblleft riempire\textquotedblright{} il template, che costituiscono il \textit{data model}.
\end{itemize}

\paragraph{\textit{Data model}}
Come parametri possono essere usate varie classi Java:

\begin{itemize}
\item \lstinline{String} per le stringhe
\item \lstinline{Number} per i numeri
\item \lstinline{Boolean} per i booleani
\item \lstinline{List} o gli array Java per sequenze di valori
\item \lstinline{Map} o bean per coppie chiave-valore (dette \textsl{hash}, come si vedrà fra poco)
\end{itemize}

Passati questi parametri, FreeMarker costruisce il \textit{data model}, che è essenzialmente un albero così formato:

\begin{itemize}
\item la radice è un nodo speciale chiamato, appunto, \textit{root}
\item i nodi intermedi, chiamati anche \textsl{variabili}, possono essere:
\begin{itemize}
\item \textit{hashes} se i loro figli sono identificati tramite nomi
\item \textit{sequences} se i figli sono identificati tramite un indice sequenziale che inizia da 0 (i.e., sono l'equivalente di un array)
\end{itemize}
In entrambi i casi, i loro figli vengono detti \textit{subvariables}
\item le foglie sono dette \textit{scalars}.
\end{itemize}

Ad esempio:

\begin{lstlisting}
(root)
  |
  +- animals
  |   |
  |   +- mouse
  |   |   |   
  |   |   +- size = "small"
  |   |   |   
  |   |   +- price = 50
  |   |
  |   +- elephant
  |   |   |   
  |   |   +- size = "large"
  |   |   |   
  |   |   +- price = 5000
  |   |
  |   +- python
  |       |   
  |       +- size = "medium"
  |       |   
  |       +- price = 4999
  |
  +- test = "It is a test"
  |
  +- whatnot
        |
        +- fruits
            |
            +- (1st) = "orange"
            |
            +- (2nd) = "banana"
\end{lstlisting}

In questo \textit{data model}:

\begin{itemize}
\item \lstinline{animals}, \lstinline{mouse}, \lstinline{elephant}, \lstinline{python} e \lstinline{whatnot} sono \textit{hashes}
\item \lstinline{fruits} è una \textit{sequence}
\item tutti gli altri nodi eccetto \textit{root} sono foglie
\item \lstinline{mouse} ha come \textit{subvariables} \lstinline{size} e \lstinline{price} (così come anche \lstinline{elephant}, ad esempio).
\end{itemize}


\paragraph{Template} 
Un file di template è un semplice file di testo (di qualunque tipo) in cui sono presenti delle variabili. Per inserire una variabile, la sintassi è:

\begin{lstlisting}
${nome}
\end{lstlisting}

Un esempio di file di template (nello specifico, si tratta di un file HTML) è il seguente:

\begin{lstlisting}
<html>
<head>
  <title>Welcome!</title>
</head>
<body>
  <h1>Welcome ${user}!</h1>
  <p>Our latest product:
  <a href="${latestProduct.url}">${latestProduct.name}</a>!
</body>
</html>
\end{lstlisting}

\paragraph{Output}
Una volta che si ha un file di template e un \textit{data model} opportuno, si cercano all'interno del modello nodi con lo stesso nome delle variabili inserite nel programma e si sostituisce il valore della variabile nel template con quello della variabile nel modello (questa sostituzione viene detta \textsl{interpolazione}). Se nel template si fa riferimento ad una proprietà di una variabile, si cerca quella proprietà fra le \textit{subvariables} della variabile nel modello. Ad esempio, considerando il template precedente ed un \textit{data model} così composto:

\begin{lstlisting}
(root)
  |
  +- user = "Big Joe"
  |
  +- latestProduct
      |
      +- url = "products/greenmouse.html"
      |
      +- name = "green mouse"  
\end{lstlisting}

l'output sarà il file:

\begin{lstlisting}
<html>
<head>
  <title>Welcome!</title>
</head>
<body>
  <h1>Welcome Big Joe!</h1>
  <p>Our latest product:
  <a href="products/greenmouse.html">green mouse</a>!
</body>
</html>
\end{lstlisting}

\subsubsection{Utilizzo}
Per utilizzare FreeMarker bisogna innanzitutto creare una configurazione tramite la classe \lstinline{Configuration}, specificando, tra le altre cose, la directory in cui si trovano i template e la codifica da adoperare.\\
In seguito, bisogna definire il modello. Per fare ciò è possibile usare diverse classi Java, a seconda del tipo di variabile da inserire; tipicamente, si utilizzano mappe o bean.\\
Infine, si crea un oggetto di tipo \lstinline{Template}, lo si inizializza con il template desiderato e si produce l'output tramite la chiamata al metodo \lstinline{process} dell'oggetto \lstinline{Template}, la cui signature è:

\begin{lstlisting}
public void process(java.lang.Object dataModel, java.io.Writer out) throws TemplateException, java.io.IOException
\end{lstlisting}

dove \lstinline{dataModel} è ovviamente il modello e \lstinline{out} è il \lstinline{Writer} sul quale produrre il risultato.\\
Di seguito è riportato un esempio di un'applicazione che esegue il processo appena descritto:

\begin{lstlisting}
import freemarker.template.*;
import java.util.*;
import java.io.*;

public class Test {

    public static void main(String[] args) throws Exception {
        
        /* ----------------------------------------------------------------------- */    
        /* You should do this ONLY ONCE in the whole application life-cycle:       */    
    
        /* Create and adjust the configuration */
        Configuration cfg = new Configuration();

        cfg.setDirectoryForTemplateLoading(new File("/where/you/store/templates"));
        cfg.setObjectWrapper(new DefaultObjectWrapper());
        cfg.setDefaultEncoding("UTF-8");
        cfg.setTemplateExceptionHandler(TemplateExceptionHandler.HTML_DEBUG_HANDLER);
        cfg.setIncompatibleImprovements(new Version(2, 3, 20));

        /* ----------------------------------------------------------------------- */    
        /* You usually do these for many times in the application life-cycle:      */    

        /* Create a data-model */
        Map root = new HashMap();
        root.put("user", "Big Joe");
        Map latest = new HashMap();
        root.put("latestProduct", latest);
        latest.put("url", "products/greenmouse.html");
        latest.put("name", "green mouse");

        /* Get the template */
        Template temp = cfg.getTemplate("test.ftl");

        /* Merge data-model with template */
        Writer out = new OutputStreamWriter(System.out);
        temp.process(root, out);
    }
}
\end{lstlisting}

