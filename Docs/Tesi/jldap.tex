\textsl{LDAP} è un protocollo per l'accesso a cartelle. Definisce un meccanismo in cui i client mandano richieste e ricevono risposte da server LDAP.
\textsl{JLDAP} è una libreria sviluppata da \textsl{Novell} che permette l'accesso al servizio di gestione di cartelle LDAP.

LDAP viene utilizzato dall'applicativo per recuperare, tramite un server LDAP di SIAF, informazioni riguardo agli utenti. In particolare consente ai docenti di ateneo di loggarsi con le proprie credenziali uniche di ateneo 
senza che ci sia bisogno di inserirli a mano in qualche database interno.

\paragraph{Funzionamento}
L'operazione centrale è la ricerca di oggetti tramite filtri di ricerca su vari attributi nel server LDAP. Il metodo che consente di effettuare questa operazione è:

\begin{lstlisting}
 LDAPSearchResults search(String base, int scope, String filter, String[] attrs, boolean typesOnly) 
\end{lstlisting}

I parametri più interessanti di questo metodo sono:
\begin{itemize}
 \item \textsl{base}\\
  Serve per specificare il punto di partenza della query, ovvero la cartella a partire dalla quale effettuare la ricerca. 
  \item \textsl{filter}\\
  Serve per specificare i criteri di ricerca: si specificano i valori di alcuni attributi dell'entità cercata.
\end{itemize}

Il risultato di questa operazione è un' entità che, per esempio, può essere rappresentata così:

\begin{lstlisting}
dn: uid=D064678,ou=docenti,ou=personale,ou=people,dc=unifi,dc=it
objectClass: unifiPersonale
cn: NOME COGNOME
gidNumber: 513
uid: D064678
uidNumber: 800064678
employeeType: Professori Associati
givenName: NOME
mail: nome.cognome@unifi.it
sn: COGNOME
userPassword:: {md5}e33ENX1ZdUkzaXNIMSthVEVFejErZUVoM1VRPT1
\end{lstlisting}

