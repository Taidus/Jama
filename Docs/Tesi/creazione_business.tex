\subsection{Edit Session e Extended Persistence Context}
Si descrive la struttura di un bean, da qui in avanti \texttt{Manager}, che possa essere usato dallo strato di presentazione per la creazione e modifica di una entità di business.

\paragraph{Caratteristiche del Manager}
\texttt{Manager} avrà un riferimento all'entità di business che stiamo creando che possa essere riempita in base ai campi riempiti dall'utente a video.
Come si è detto la creazione/modifica di una entità di business in generale è realizzata attraverso più passi di una procedura, e quindi non può essere confinata in una sola request. Questo suggerisce che \texttt{Manager} debba essere 
\textsl{reques scoped}. Si sottolinea che non sarebbe possibile usare un bean di tipo \textsl{session scoped} perchè questo comporterebbe la condivisione del bean fra due tab del browser: l'utente che crede di creare due entità in parallelo
sta invece modificando la stessa!

Direttamente collegato alla questione dello scope c'è il problema del detachment: se il Persistence Context che viene usato ha il proprio ciclo di vita legato alla transazione, necessariamente dopo il recupero dal database l'entità diventerà
detached. Come si è spiegato in \ref{jpa} lo stato di un entità detached non verrà scritto sul database in nessuna transazione. Sebbene sia possibile riportare un' entità da detached a managed tramite
l'operazione di \texttt{merge()}, in generale è preferibile non farlo perchè la gestione dei riferimenti dell'entità è problematica. Per ovviare a questo problema possiamo optare per un Entity Manager di tipo extended. Questo ci garantisce
che durante tutto il ciclo di vita del bean \texttt{Manger}, ovvero per tutta la conversation, avremmo un solo Persistence Context, e di conseguenza l'entità che stiamo modificando non sarà mai detached.

Un'altra problematica che si deve affrontare è prevedere la possibilità di annullare la creazione/modifica in qualsiasi momento della procedura. Per risolvere questo problema dobbiamo gestire le transazioni del container.
Un modo elegante per farlo è annotare il bean con l'annotazione 
\begin{lstlisting} 
@TransactionAttribute(TransactionAttributeType.NOT_SUPPORTED)
\end{lstlisting}
e il metodo che conclude la procedura salvando i dati con 
\begin{lstlisting}
@TransactionAttribute(TransactionAttributeType.REQUIRES_NEW)
\end{lstlisting}



Il bean \texttt{Manager} può quindi essere strutturato come segue:

\begin{lstlisting}
 
@ConversationScoped
@Stateful
@TransactionAttribute(TransactionAttributeType.NOT_SUPPORTED)
public class Manager implements Serializable {

	private static final long serialVersionUID = -4966124878956728047L;
	@Inject
	private Conversation conversation;

	private Entity entity;

	@PersistenceContext(unitName = "primary", type = PersistenceContextType.EXTENDED)
	private EntityManager em;


	public UserEditorBean() {
		super();
	}


	private void begin() {

		conversation.begin();
	}


	@Remove
	private void close() {

		conversation.end();

	}


	@TransactionAttribute(TransactionAttributeType.REQUIRES_NEW)
	public String save() {
		
		em.persist(entity);
		
		close();

		return "home";
	}


	public String cancel() {
		close();
		return "home";
	}


	
	public String createUser() {
		begin();
		currentUser = new User();
		return "wizard";
	}


	public String editEntity( int id) {

		begin();

		entity = em.find(id);
		return "wizard";
	}


	}

}
\end{lstlisting}




