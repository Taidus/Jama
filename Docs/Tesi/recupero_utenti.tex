La classe che si occupa del recupero degli utenti è \texttt{UserDaoBean}: è possibile recuperare gli utenti sia dal database interno sia sfruttando il servizio LDAP in modo del tutto trasparente.
Il database ha priorità rispetto ad LDAP in modo che sia possibile sovrascrivere alcune informazioni sugli utenti semplicemente inserendoli nel database. Per quanto riguarda il recupero dal database è sufficiente una query per matricola 
mentre per ottenere i dati dal server LDAP è necessario effettuare qualche operazione in più, perciò il codice che compie questa operazione è stato incapsulato in una classe chiamata \texttt{LdapManager}.

Tale classe effettua la connesione al server LDAP tramite il protoccolo sicuro \textsl{ldaps} e successivamente sfrutta il metodo \texttt{search(...)} come visto in INSERIRE RIFERIMENTO per ottenere i dati sull'utente in base alla sua matricola.
Tale ricerca è effettuata impostando il parametro \textsl{base} a ``ou=people, dc=dinfo, dc=unifi, dc=it''e \textsl{filter} a ``uid= <matricola> ''. Il parametro base viene letto dal file di configurazione 
\path{standalone/deployments/Jama.war/WEB-INF/classes/config/ldap.properties}, quindi è possibile cambiarlo in qualsiasi momento se lo schema esposto dal server LDAP dovesse cambiare.
A partire dai dati contenuti nel \texttt{LDAPSearchResults} restituito da \texttt{search(...)} è possibile istanziare la classe \texttt{User} che rappresenta un utente.

Il metodo di \texttt{UserDaoBean} che restituisce un utente in base alla matricola è riportato qui sotto.

\begin{lstlisting}
public User getBySerialNumber(String serialNumber) {
	User result = null;
	
	try {
		result = em.createNamedQuery("User.findBySerialNumber", User.class).setParameter("number", serialNumber).getSingleResult();
		
		if (null == result) {
			result = ldapm.getUser(serialNumber);
		}
		
	} catch (NoResultException e) {
		result = null;
	}

	return result;

}
\end{lstlisting}



