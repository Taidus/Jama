Di seguito elenchiamo i principali casi d'uso per ciascun tipo di utente che interagisce col sistema.

\begin{itemize}
 \item Operatore
    \begin{enumerate}
     \item Inserimento di una nuova Convenzione/Contributo\newline
     
     Percorso base:
	L'operatore una volta effettuato il login clicca su ``Crea una Convenzione/Contributo''; viene visualizzata una schermata suddivisa in varie schede, ognuna corrispondente ad un passo della procedura. E' possibile passare da una
	vista all'altra mediante i pulsanti "Avanti" e "Indietro". I passi sono:
	\begin{enumerate}
	 \item Inserimento dei dati della convenzione\newline
	    Viene presentata una schermata in cui sono elencati tutti i campi necessari per la definizione di una convenzione/contributo, 
	    che l'Operatore deve compilare. Tali campi sono:
	    \begin{itemize}
	     \item Il titolo
	     \item Il responsabile scientifico\newline 
		Per riempire questo campo, bisogna scegliere il responsabile dal rispettivo menù a tendina. 
		Se l'utente non è presente all'interno della lista, è possibile crearne uno nuovo cliccando su "Aggiungi".
	     \item ...
	    \end{itemize}
	    
	 \item Inserimento della tabella di ripartizione\newline
	    Viene presentata una schermata in cui sono presenti le voci della tabella di ripartizione. L'Amministratore inserisce, per ogni voce, i valori percentuali 
			in base ai quali dividere l'importo totale. Il sistema mostra i valori così calcolati. I dati
			da inserire sono:
	    \begin{itemize}
	     \item ...
	     \item L'ultimo campo, "Beni e servizi", non è modificabile dall'utente: viene calcolato automaticamente per differenza.
		Un procedimento del tutto analogo deve essere seguito per la suddivisione dell'importo relativo a "Beni e servizi": l'Amministratore inserisce i valori desiderati
		ad eccezione dell'ultimo, calcolato per differenza.
	    \end{itemize}
	 \item Inserimento della documentazione relativa alla convenzione.
		Viene presentata una schermata in cui appaiono i documenti allegati alla convenzione. L'Amministratore può aggiungere o eliminare un documento 
		cliccando sugli appositi tasti. Premendo il tasto "Salva" la convenzione viene salvata e la procedura termina. Si ritorna alla schermata precedente.
	\end{enumerate}
		 
			
				
		

Percorso alternativo:
	Durante uno qualsiasi dei passi, l'Amministratore può cliccare il tasto "Annulla", che comporta, a seguito di una conferma, il ritorno alla schermata precedente
	senza che la convenzione venga inserita o i cambiamenti effettuati salvati.

	Se l'Amministratore clicca sul tasto "Salva" senza aver compilato dei campi obbligatori, o avendo inserito dei valori non consentiti, viene visualizzato un messaggio di errore 
	e il documento non viene salvato. La schermata non viene cambiata, in modo che l'Amministratore possa procedere alla correzione.
			

     
     
     
     
     
     
     
    \end{enumerate}

 
 
 \item Docente
 \item Amministratore
\end{itemize}
