Di seguito elenchiamo i principali casi d'uso per ciascun tipo di utente che interagisce col sistema.
\paragraph{Operatore}
\begin{enumerate}
  \item Inserimento di una nuova Convenzione/Contributo\\
  
  Percorso base:
  L'operatore una volta effettuato il login clicca su ``Crea una convenzione/contributo''; viene visualizzata una schermata suddivisa in varie schede,
  ognuna corrispondente ad un passo della procedura. E' possibile passare da una vista all'altra mediante i pulsanti ``Avanti'' e ``Indietro''. I passi sono:
  \begin{enumerate}
    \item Inserimento dei dati della convenzione/contributo\\
      
      Viene presentata una schermata in cui sono elencati tutti i campi necessari per la definizione di una convenzione/contributo, 
      che l'Operatore deve compilare. Tali campi sono:
      \begin{itemize}
	\item Il titolo
	\item Il numero di protocollo
	\item L'UAR
	\item La tipologia
	\item Il responsabile scientifico\\
	  Per riempire questo campo, bisogna scegliere il responsabile dal rispettivo menù a tendina. 
	  Se l'utente non è presente all'interno della lista, è possibile crearne uno nuovo cliccando su "Aggiungi".
	\item Il referente
	\item La dittà\\
	   Per riempire questo campo, bisogna scegliere il responsabile dal rispettivo menù a tendina. 
	  Se l'utente non è presente all'interno della lista, è possibile crearne uno nuovo cliccando su "Aggiungi".
	\item Il nome del progetto CIA
	\item Il Repertorio
	\item Il totale imponibile
	\item L'Iva
	\item La data di approvazione
	\item La data di inizio
	\item La data di scadenza
      \end{itemize}
      
    \item Inserimento della tabella di ripartizione\\
     
      Viene presentata una schermata in cui sono presenti le voci della tabella di ripartizione. L'Operatore può modificare alcuni valori percentuali 
      in base ai quali dividere l'importo totale. Le voci non modificabili sono calcolate in relazione ai campi modificabili facendo riferimento alle norme di ateneo. Le voci modificabili sono:
      \begin{itemize}
	\item ``Personale'': stabilisce la quota destinata al personale; è l'unico campo principale che l'operatore può modificare e in base al quale vengono calcolati gli altri.
	\item ``Missioni'', ``Materiale di consumo'', etc. sono sottocampi di ``Beni e Servizi'' e servono per meglio specificare come verrà ripartita la quota destinata a ``Beni e Servizi''.
      \end{itemize}
    \item Gestione delle rate\\
      
      Questo passo della procedura è facoltativo: viene data la possibilità all'operatore di inserire delle rate per la convenzione/contributo
      che sta creando. Per i dettagli riquardo all'inserimento della rata si rimanda al corrispettivo caso d'uso. Una volta inserite una o più rate
      queste vengono visualizzate in una tabella e l'Operatore
      ha la possibilità di:
      \begin{itemize}
       \item  Modificare o visualizzare una rata; cliccando sui tasti ``Modifica'' o ``Visualizza'' che compaiono sulla destra, nella riga della tabella
       corrispondente alla rata in questione, viene mostrata la schermata "Modifica/Visualizzazione di una rata"; questa è analoga a quella relativa all'inserimento
       eccetto per il fatto che i campi	sono inizializzati al valore correntemente assegnato; nel caso della visualizzazione i campi non sono modificabili.
			
	\item Eliminare una rata cliccando sul tasto ``Elimina''. Viene chiesta conferma all'Operatore mediante una finestra di dialogo.

      \end{itemize}

      
    \item Inserimento della documentazione relativa alla convenzione\\
	  
	  Viene presentata una schermata in cui appaiono i documenti allegati alla convenzione. L'Operatore può aggiungere o eliminare un documento 
	  cliccando sugli appositi tasti. Premendo il tasto ``Salva'' la convenzione viene salvata e la procedura termina. Si ritorna alla schermata precedente.
  \end{enumerate}
		 
  Percorso alternativo:
  Durante uno qualsiasi dei passi, l'Operatore può cliccare il tasto ``Annulla'', che comporta, a seguito di una conferma, il ritorno alla schermata precedente
  senza che la convenzione/contributo venga inserita o i cambiamenti effettuati salvati.
  Se l'Operatore clicca sul tasto ``Salva'' senza aver compilato dei campi obbligatori, o avendo inserito dei valori non consentiti, viene visualizzato un messaggio di errore 
  e il documento non viene salvato. La schermata non viene cambiata, in modo che l'Operatore possa procedere alla correzione.
     
  \item Visualizzazione delle convenzioni/contributi\\

  Percorso base:
  L'operatore clicca su ``Visualizza la lista delle convenzioni/contributi''; viene mostrata una lista delle convenzioni/contributi correntemente stipulate in 
  riferimento al dipartimento di afferenza dell'Operatore, con opportuni filtri 
  per agevolare la ricerca (tra cui un filtro per data di scadenza) e ordinabili secondo la data.
  L'Operatore può selezionare una convenzione e compiere 3 azioni mediante gli appositi pulsanti, che appaiono sulla destra portando il mouse su 
  una convenzione:
  \begin{itemize}
   \item visualizzarla;
   \item modificarla (si rimanda al rispettivo caso d'uso);
   \item eliminarla.
  \end{itemize}		

  Percorso alternativo:
  Si può tornare alla Home con l'apposito tasto.
 
  \item Modifica di una convenzione\\
  
  Percorso base:
  L'Operatore a partire dalla schermata ``Lista delle Convezioni/Contributi'' clicca sul pulsante ``Modifica'' che appare sulla destra nella riga 
  della tabella corrispondente alla convenzione/contributo desiderata. Viene visualizzata una schermata suddivisa in schede del tutto analoga a quella
  descritta nel caso di creazione di una convenzione/contributo con la differenza che in questo caso è possibile spostarsi da una scheda all'altra
  cliccando sulla scheda stessa. Inoltre è presente la scheda aggiuntiva ``Riepilogo'' che contiene alcune informazioni di riepilogo come il residuo
  totale della convenzione/contributo o il totale fatturato.
  L'Operatore effettua i cambiamenti desiderati e clicca sul pulsante ``Salva''.\\

  Percorso alternativo:
  L'Operatore può cliccare sul tasto "Annulla" in qualsiasi momento per tornare alla schermata "Visualizzazione delle convenzioni" senza salvare le modifiche effettuate.
  Se l'Operatore clicca su "Salva" ma alcuni valori immessi non sono corretti, viene visualizzato un messaggio di errore e non si torna alla schermata "Visualizzazione delle
  convenzioni". I cambiamenti effettuati non vengono (ovviamente) salvati.
  
\item Visualizzazione di una convenzione/contributo\\
 
  Del tutto analogo a ``Modifica di una convenzione/contributo'' con la differenza che in questo caso non è possibile modificare i dati della
  convenzione/contributo.
  
  
\item Inserimento di una rata\\
L'operatore può inerire una rata sia in fase di creazione della convenzione/contributo sia in fase di modifica; in entrambi i casi dopo aver raggiunto
la scheda ``Rate'' l'Operatore clicca sul pulsante ``Aggiungi una rata''.  
Viene visualizzata una finestra di dialogo suddivisa in varie schede,
ognuna corrispondente ad un passo della procedura. E' possibile passare da una vista all'altra mediante i pulsanti "Avanti" e "Indietro". I passi sono:
\begin{enumerate}
  \item Inserimento dei dati della rata\\
  
  L'operatore inserisce i seguenti campi
    \begin{itemize}
    \item ...
    \end{itemize}
  \item Inserimento della tabella di ripartizione
  Il procedimento è del tutto analogo  a quello descritto nel caso d'uso "Inserimento della tabella di ripartizione" per l'inserimento di 
  una nuova convenzione/contributo. 
\end{enumerate}

Le modifiche vengono salvate cliccando sul pulsante "Salva". Si ritorna alla schermata precedente;


\end{enumerate}

 
\paragraph{Docente}

\begin{enumerate}
 
 \item Visualizzazione della lista delle convenzioni/contributi\\
 
 Il docente, dopo aver effettuato il login, può cliccare sul pulsante ``Visualizzazione della lista delle convenzioni/contributi''; la schermata
 che viene visualizzata contiene una tabella che elenca le convenzioni/contributi del docente. E' possibile filtrare le convenzioni/contributi
 secondo vari criteri(data, tipo, scadenze più vicine, ...).Inoltre è possibile visualizzare i dettagli di una convenzione/contributo cliccando sul
 pulsante ``Visualizza'' che appare posizionando il puntatore su una riga della tabella.
 
 \item Visualizzazione di una convenzione/contributo di cui il docente è responsabile scientifico\\
 Il docente dalla schermata ``Lista delle convenzioni/contributi'' può cliccare sul pulsante ``Visualizza'' relativo ad una convenzione/contributo; compare una schermata suddivisa in schede analoga a quella della modifica/creazione
 della convenzione. Il docente può navigare fra le schede cliccandoci sopra. Non è permessa nessuna modifica ai dati della convenzione/contributo tuttavia il docente può inserire degli allegati dalla scheda ``Allegati''.Cliccando su ``Salva''
 gli allegati inseriti dal docente vengono memorizzati, al contrario cliccando su ``Indietro'' le modifiche vegono scartate .
 
 
 
\end{enumerate}



\paragraph{Amministratore}
\begin{enumerate}
 \item Inserimento di un nuovo utente
 \item Visualizzazione della lista degli utenti
\end{enumerate}

\paragraph{Tempo}
\begin{enumerate}
 \item Notifica delle scadenze
\end{enumerate}




