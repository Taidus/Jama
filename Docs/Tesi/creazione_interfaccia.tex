
Il processo di creazione di un entità del modello coinvolge vari componenti:

\begin{itemize}
\item una pagina web che viene utilizzata nella creazione dell'interfaccia grafica
\item (opzionale) un bean di presentazione, che garantisce un corretto funzionamento della pagina
\item un controllore, che fa da intermediario tra il livello di presentazione e il modello di business
\item un DAO, che si occupa della comunicazione con il database
\end{itemize}

\subsection{Livello presentazione}

Quando una schermata di creazione viene visualizzata, per prima cosa viene istanziato dal \textit{container} il controllore che si occupa della creazione dell'entità, il quale a sua volta inizia una conversazione e crea l'oggetto che conterrà i valori inseriti tramite interfaccia dall'utente.\\
Graficamente le pagine web di questo tipo sono costituite semplicemente da alcuni campi di input in cui inserire i dati, inseriti all'interno di un wizard se il numero di valori da inserire è troppo elevato e renderebbe troppo carica un'unica schermata; in questo caso, si rende a volte necessario un bean di presentazione per controllare il flusso del wizard. Normalmente, i form non sono inseriti direttamente nella pagina; piuttosto, si utilizzano vari componenti personalizzati, i quali vengono inclusi nella pagina per ottenere l'effetto desiderato. In questo modo, lo stesso componente può essere utilizzato in più contesti. Questo è reso possibile grazie al tag \lstinline{composite} di JSF, che consente di creare componenti ed inserirle all'interno di un namespace personalizzato.\\
La pagina è poi ovviamente corredata di pulsanti per salvare i dati inseriti o per tornare alla vista precedente ignorando i cambiamento. Nel caso in cui l'utente decida di salvare le modifiche effettuate, il controllore provvede ad aggiornare il database per mezzo del DAO opportuno. In entrambi i casi, la conversazione viene chiusa e si ritorna alla schermata precedente.