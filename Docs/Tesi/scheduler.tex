Si è annoverato insieme agli agenti che interagiscono con il sistema anche il Tempo. In questa sezione vediamo come si è realizzato questo tipo di agente, e come si sono implementate le azioni che esso compie.
In particolare si analizzano quelle azioni che possono essere classificate come task periodici e che non possono essere effettuate come conseguenza di un azione di un utente come l'Operatore, il Docente o l'Amministratore.
Un buon candidato è la notifica delle scadenze più vicine: ogni giorno ad un' ora fissata è necessario cercare le scadenze da notificare e spedire una email al docente di riferimento.

Si è scelto di rappresentare questa operazione sfruttando la classe \texttt{java.util.TimerTask}. Per definire un task occorre quindi estendere la classe \texttt{TimerTask} ed implementare il metodo \texttt{run()} come è mostrato di seguito.
\begin{lstlisting}
 
public class MyTask extends TimerTask {
	...

	@Override
	public void run() {
	
	  dostuff();

	}

}
 
\end{lstlisting}

Una volta definita l'operazione occorre schedulare la sua esecuzione impostando un tempo di partenza e un intervallo di ripetizione. Possiamo fare ciò utilizzando la classe \texttt{java.util.Timer} come mostrato di seguito.

\begin{lstlisting}
		Date date = ...;

		Timer timer = new Timer();
		TimerTask task = new MyTask();

		long interval = ...;
		timer.schedule(task, date, period);
 
\end{lstlisting}

L'ultimo problema che è necessario affrontare è come poter eseguire il codice sopra riportato all'avvio dell'\textit{application server}. La soluzione che si propone è utilizzare un bean che venga creato allo startup del server, utilizzando l'annotazione
 \texttt{@Startup}, e che contenga tale
codice in un metodo annotato con \texttt{@PostConstruct}. Un metodo così annotato verrà eseguito dopo il costruttore del bean ma prima di ogni altro metodo. Il risultato è quindi un bean così fatto:

\begin{lstlisting}
@ApplicationScoped
@Singleton
@Startup
@Named("scheduler")
public class Scheduler {

	public Scheduler() {}


	@PostConstruct
	public void schedule() {

		Date date = ...;

		Timer timer = new Timer();
		TimerTask task = new MyTask();

		long interval = ...;
		timer.schedule(task, date, period);

	}

}
\end{lstlisting}
