\section{Tecnologie}

\subsection{Introduzione}

\begin{frame}{Tecnologie utilizzate}

\begin{itemize}
\item JBoss AS 7.1
	\begin{itemize}
	\item fornisce l'infrastruttura per consentire l'esecuzione di una web application
	\end{itemize}

\item Java EE 6
	\begin{itemize}
	\item piattaforma software per lo sviluppo di applicazioni enterprise
	\item comprende varie tecnologie, fra cui:
		\begin{itemize}
		\item JPA (Java Persistence API) 2.0
		\item CDI (Contexts and Dependency Injection) 1.0
		\item JSF (JavaServer Faces) 2.0
		\end{itemize}
		
	\end{itemize}

\end{itemize}


\end{frame}

\begin{frame}{Bean}

\begin{itemize}
\item Sono componenti software riusabili che possono essere gestiti dal container

\item Vari tipi:
	\begin{itemize}
	
	\item \textsl{Enterprise JavaBeans} (EJB): bean utilizzati per la logica di business o la persistenza
		\begin{itemize}
		\item session bean
		\item message-driven bean
		\item entity bean
		\end{itemize}
		
	\item  \textsl{Managed beans}: bean utilizzati a livello di presentazione
	\end{itemize}

\end{itemize}

\end{frame}


\subsection{JPA}
\begin{frame}{Java Persistence API}

\begin{itemize}
\item Si occupa della persistenza dei dati
\end{itemize}

\begin{itemize}
\item Problema:

	\begin{itemize}
	\item Nelle applicazioni enterprise è necessario ricorrere ad un database
	
	\item Il modello più utilizzato all'interno dei database è il \textsl{modello relazionale}
		\begin{itemize}
		\item le entità sono le \textsl{relazioni}
		\item sono collegate tra di loro mediante \textsl{chiavi}
		\end{itemize}
	
	\item Il modello utilizzato all'interno delle applicazioni Java è il \textsl{modello a oggetti}
		\begin{itemize}
		\item le entità sono le \textsl{classi}
		\item sono collegate tra di loro mediante \textsl{riferimenti}
		\end{itemize}
	
	\end{itemize}

\end{itemize}

\end{frame}


\begin{frame}{Integrazione fra i modelli}

\begin{itemize}
\item Occorre un sistema per la conversione tra i due modelli

	\begin{itemize}
	\item non deve influire pesantemente sullo sviluppo dell'applicazione
	\item in particolare:
	
		\begin{itemize}
		\item le classi del modello a oggetti non devono essere obbligate ad estendere particolari classi per poter essere persistite
		\item deve essere possibile recuperare informazioni senza dover scrivere espressioni che coinvolgano costrutti tipici di un database
		\item deve essere possibile integrarsi con database preesistenti
		\end{itemize}
	
	\end{itemize}

\end{itemize}

\end{frame}


\begin{frame}{Costrutti fondamentali}

\begin{itemize}
\item Entità
	\begin{itemize}
	\item è un'unità che possiede uno stato e può essere persistita
	\item una classe Java è un'entità se:
		\begin{itemize}
		\item è annotata \texttt{@Entity}
		\item ha un costruttore senza parametri
		\end{itemize}
	\end{itemize}

\item Entity Manager
	\begin{itemize}
	\item gestisce le entità
	\item è associato ad un \textsl{Persistence Context}
		\begin{itemize}
		\item è l'insieme delle entità gestite da un Entity Manager in un dato momento
		\item se un entità fa parte di un Persistence Context viene detta \textsl{managed}
		\item altrimenti viene detta \textsl{detached}
		\end{itemize}
	\end{itemize}

\end{itemize}	


\end{frame}

\begin{frame}{Query}

\begin{itemize}
\item Si usa il \textsl{Java Persistence Query Language} (JPQL)

\item Sintassi simile a SQL, ma:

	\begin{itemize}
	\item è indipendente dal database
	\item utilizza le entità del modello di dominio e relativi attributi
	\end{itemize}


\end{itemize}

\end{frame}


\subsection{CDI}

\begin{frame}{Contexts and Dependency Injection}

\begin{itemize}
\item Problema: incompatibilità tra
	\begin{itemize}
	\item il livello di business, che usa EJB;
	\item il livello di presentazione, che usa Managed bean.
	\end{itemize}
	
\item CDI:
	\begin{itemize}
	\item consente l'utilizzo di EJB nel livello di presentazione
	\item fornisce una specifica per la definizione di Managed bean
	\end{itemize}
\end{itemize}

\end{frame}



\begin{frame}{Panoramica}

\begin{itemize}
\item Servizi offerti:
	\begin{itemize}
	\item Contesto
	\item Iniezione di dipendenza
	\end{itemize}

\item Caratteristiche principali:
	\begin{itemize}
	\item integrazione Expression Language
	\item disaccoppiamento
	\item controllo sui tipi
	\end{itemize}
\end{itemize}

\end{frame}

\begin{frame}{Bean CDI}

\begin{itemize}
\item \textsl{Scope}
	\begin{itemize}
	\item lega il bean ad un particolare contesto
	\item determina il ciclo di vita
	\item limita la visibilità al client
	\end{itemize}
	
\item 4 possibili scope:
	\begin{itemize}
	\item Request
	\item Conversation
	\item Session
	\item Application
	\end{itemize}

\end{itemize}

\end{frame}



\subsection{JSF}

\begin{frame}{JavaServer Faces}

\begin{itemize}
\item Framework per lo sviluppo di interfaccia grafica
\item \textsl{Component-based}
\item Consente estensione delle funzionalità tramite librerie di terze parti
\end{itemize}

\end{frame}


\begin{frame}{Architettura}

\begin{itemize}
\item Architettura Model-View-Controller
\end{itemize}

\begin{figure}
	\centering
	\includegraphics{JSF_architecture.png}
\end{figure}

\end{frame}


\begin{frame}{Comunicazione client-server}

\begin{enumerate}
\item \textsl{Encoding}
\item invio pagina al client
\item il client naviga ed invia dati al server
\item \textsl{decoding}
\item il server processa la richiesta
\end{enumerate}

\begin{figure}
	\centering
	\includegraphics{JSF_client-server_communication.png}
\end{figure}

\end{frame}

\begin{frame}{Ciclo di vita}



\end{frame}






















